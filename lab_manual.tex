%%%%%%%%%%%%%%%%%%%%%%%%%%%%%%%%%%%%%%%%%
% Tufte-Style Book (Minimal Template)
% LaTeX Template
% Version 1.0 (5/1/13)
%
% This template has been downloaded from:
% http://www.LaTeXTemplates.com
%
% License:
% CC BY-NC-SA 3.0 (http://creativecommons.org/licenses/by-nc-sa/3.0/)
%
% IMPORTANT NOTE:
% In addition to running BibTeX to compile the reference list from the .bib
% file, you will need to run MakeIndex to compile the index at the end of the
% document.
%
%%%%%%%%%%%%%%%%%%%%%%%%%%%%%%%%%%%%%%%%%

%----------------------------------------------------------------------------------------
%	PACKAGES AND OTHER DOCUMENT CONFIGURATIONS
%----------------------------------------------------------------------------------------

\documentclass{tufte-book} % Use the tufte-book class which in turn
                           % uses the tufte-common class

\definecolor{dartmouthgreen}{RGB}{0, 112, 60}

\hypersetup{colorlinks=true,linkcolor=dartmouthgreen} % Comment this line if you don't wish to have colored links

\usepackage{microtype} % Improves character and word spacing

%\usepackage{lipsum} % Inserts dummy text

\usepackage{booktabs} % Better horizontal rules in tables

\usepackage{graphicx} % Needed to insert images into the document
\graphicspath{{graphics/}} % Sets the default location of pictures
\setkeys{Gin}{width=\linewidth,totalheight=\textheight,keepaspectratio} % Improves figure scaling

\usepackage{fancyvrb} % Allows customization of verbatim environments
\fvset{fontsize=\normalsize} % The font size of all verbatim text can be changed here

\newcommand{\hangp}[1]{\makebox[0pt][r]{(}#1\makebox[0pt][l]{)}} % New command to create parentheses around text in tables which take up no horizontal space - this improves column spacing
\newcommand{\hangstar}{\makebox[0pt][l]{*}} % New command to create asterisks in tables which take up no horizontal space - this improves column spacing

\usepackage{xspace} % Used for printing a trailing space better than
                    % using a tilde (~) using the \xspace command

\usepackage{hyperref} %web links/URLs

\newcommand{\monthyear}{\ifcase\month\or January\or February\or March\or April\or May\or June\or July\or August\or September\or October\or November\or December\fi\space\number\year} % A command to print the current month and year

\newcommand{\openepigraph}[2]{ % This block sets up a command for printing an epigraph with 2 arguments - the quote and the author
\begin{fullwidth}
\sffamily\large
\begin{doublespace}
\noindent\allcaps{#1}\\ % The quote
\noindent\allcaps{#2} % The author
\end{doublespace}
\end{fullwidth}
}

\newcommand{\ourschool}{Dartmouth College}

\newcommand{\blankpage}{\newpage\hbox{}\thispagestyle{empty}\newpage} % Command to insert a blank page

\usepackage{makeidx} % Used to generate the index
\makeindex % Generate the index which is printed at the end of the document

%----------------------------------------------------------------------------------------
%	BOOK META-INFORMATION
%----------------------------------------------------------------------------------------

\title{Lab Manual} % Title of the book

\author{Jeremy R. Manning, Ph.D.} % Author

\publisher{Contextual Dynamics Lab, \ourschool} % Publisher

%----------------------------------------------------------------------------------------

\begin{document}

\frontmatter

%----------------------------------------------------------------------------------------
%	EPIGRAPH
%----------------------------------------------------------------------------------------

\thispagestyle{empty}
%\openepigraph{Quotation 1}{Author, {\itshape Source}}
%\vfill
%\openepigraph{Quotation 2}{Author}
%\vfill
%\openepigraph{Quotation 3}{Author}

%----------------------------------------------------------------------------------------

\maketitle % Print the title page

%----------------------------------------------------------------------------------------
%	COPYRIGHT PAGE
%----------------------------------------------------------------------------------------

\newpage
\begin{fullwidth}
~\vfill
\thispagestyle{empty}
\setlength{\parindent}{0pt}
\setlength{\parskip}{\baselineskip}
Copyright \copyright\ \the\year\ \thanklessauthor

\par\smallcaps{Published by the \thanklesspublisher}

\par\smallcaps{\url{http://caligari.dartmouth.edu/\~jmanning}}

%\par License information.\index{license}

\par\textit{First printing, \monthyear}
\end{fullwidth}

%----------------------------------------------------------------------------------------

\setcounter{tocdepth}{1}
\tableofcontents % Print the table of contents

%----------------------------------------------------------------------------------------

%\listoffigures % Print a list of figures

%----------------------------------------------------------------------------------------

%\listoftables % Print a list of tables

%----------------------------------------------------------------------------------------
%	DEDICATION PAGE
%----------------------------------------------------------------------------------------

% \cleardoublepage
% ~\vfill
% \begin{doublespace}
% \noindent\fontsize{18}{22}\selectfont\itshape
% \nohyphenation
% Dedicated to my family and friends.
% \end{doublespace}
% \vfill
% \vfill

%----------------------------------------------------------------------------------------
%	INTRODUCTION
%----------------------------------------------------------------------------------------

\cleardoublepage
\chapter{Introduction}\label{ch:intro} % Adding an asterisk leaves out this chapter from the table of contents
This lab manual is intended to serve two purposes.  First, the manual
provides a comprehensive overview of official lab policies,
expectations, facilities, and personnel.  Second, it provides a set of
general tutorials and a list of relevant links, pointers, and/or
references related to the techniques we employ in our research.

\newthought{Who is this lab manual for?}  \marginnote{\texttt{TASK:}
  Upon reading through this lab manual for the first time, please make
  at least one edit.  You could correct a typo, clarify something
  that's unclear, add a comment, etc.  Focus your edits or additions
  on sections that are most relevant to the work you want to do.
  Importantly, be sure to push your edits to the manual's github
  repository so that everyone can benefit.} Every new lab member
should read the latest version of this lab manual in detail and
reference it later as needed.  Periodically throughout the document,
you will see margin notes with listed \texttt{TASK} items.  Completing
your read through entails both reading the contents of the manual and
completing the relevant \texttt{TASK} items.

\newthought{What should you do if you don't understand something?}
\marginnote{\texttt{TASK:} If you don't understand something, ask
  another lab member for help!}If you don't understand something you
read in this manual, it is important that you \textit{ask another lab
  member for help}.  Every member of the lab brings their own unique
knowledge base, training, life experiences, and perspectives.
Respecting and celebrating those differences drives the science we do.
If you're new to the lab or new to a particular technique, you might
feel like a newbie today--- but chances are good that if you stick
around for a bit someone else will be seeking your expert opinion
before you know it.  In addition to learning, there's another good
reason for asking for help: if you don't understand something you read
in this manual, there's a reasonable chance that you've discovered a
mistake!

\newthought{Why is it worth my time to read through the manual?}
Aside from pursuing your own curiosity, a major reason that you've
decided to join an academic research lab is probably because you want
to gain training or career-advancing experiences.  This manual
summarizes the collective wisdom of past and present lab members in a
way that we think will best allow you to achieve your objectives.
\textit{Learn from it}, \textit{challenge it}, and \textit{add to it}.



\chapter{Official lab policies}\label{ch:policy}
Our lab's policies are intended to provide a framework for
\textit{maximizing efficiency}.  Achieving our peak efficiency as a
lab means we are being as scientifically productive as possible, in terms of knowledge
discovery (learning new stuff) and disemination (papers, talks,
conference presentations, publicly released datasets, software,
etc.). It also means that our fellow lab members are achieving their
training and career objectives.  To achieve peak efficiency we need to
succeed on two fronts:
\begin{itemize}
\item \textbf{Communication.}  We want to foster an environment where
  everyone feels comfortable contributing to the collective dialogue.
  Our lab meets regularly to discuss logistical (e.g.\ temporal, financial,
  sociological) and technical issues.  We also use a variety of
  software packages to synchronize and facilitate communication within
  our lab and between our lab and the broader scientific community.
\item \textbf{Resource allocation.}  Our lab resources (e.g.\
  equipment, time, money, attention) are finite.  We want to foster an
  environment where lab resources are used as efficiently as possible
  to achieve our collective goals.
\end{itemize}
Your job as a contributing lab member is to help us to achieve our 
collective peak efficiency (as a lab) while also maximizing your own training and
career potential.

\newpage
\section{The lab hierarchy}
Our lab is organized in a roughly hierarchical structure.  The
hierarchy is an organizational tool that helps to define lab policies
and expectations. Importantly, the lab hierarchy does \textit{not}
serve an excuse for disrespecting or discounting the opinions of lab
members at similar or different levels of the hierarchy.  Each lab
member's position in this hierarchy is determined by two factors: the
lab member's job title and the amount of time they have worked in the
lab.  Moving ``up'' in the hierarchy generally entails working in the
lab for some amount of time, gaining experience by working in other
(academic or industrial)
labs, and/or earning academic degrees.  The lab hierarchy is intended
to serve as a general framework for estimating what is expected of
each lab member (e.g.\ in terms of research, supervising and
mentoring, training roles, and other lab responsibilities).  The
hierarchy also serves as a general framework for determining lab
member salaries and benefits.  There are two branches of the lab
hierarchy; although personnel on different branches may have
overlapping responsibilities, the first branch is primarly concerned
with directly carrying out lab research and the second branch is
primariy concerned with supporting or managing the lab's research
objectives.  The research-focused levels of the lab hierarchy are
defined as follows: \newcounter{levels}
\begin{list}{R\arabic{levels}-$x$:~}{\usecounter{levels}}
\item \textbf{Undergraduate research
    assistants.}\marginnote{Research-focused levels of the lab hierarchy}  This category
  includes undergraduate students (currently enrolled at \ourschool)
  who are pursing an active undergraduate research program in the lab.
  An active undergraduate research program may include for-credit
  projects (such as an independent study or an honors thesis project)
  or not-for-credit projects.
\item \textbf{Postbaccalaureate research assistants.}  This category
  includes lab members who have already earned an undergraduate degree
  (e.g.\ BA, BS) but who have not earned a graduate degree, and who
  are not currently enrolled in a degree-granting graduate program.
\item \textbf{Postgraduate research assistants.}  This category
  includes lab members who have already earned a non-doctorate
  graduate degree (e.g.\ MA, MS) but who are not currently enrolled in
  a degree-granting graduate program.
\item \textbf{Graduate students.}  This category includes lab memebers
  who are currently enrolled in a degree-granting graduate program
  (generally working towards a master's degree or doctorate).
\item \textbf{Postdoctoral researchers.}  This category includes lab
  members who have earned a doctorate degree and are not currently
  enrolled in a degree-granting program.
% \item \textbf{Principle investigators.}  This category includes lab
%   members (at any level or designation) who have successfully obtained external funding for an
%   independent research or training project, and whose funding is
%   currently active.  If you secure your own research funding, you are
%   granted a corresponding increase in the level of control over and
%   responsibility for your project-related lab duties.
\end{list}
Note that the $x$'s above should be replaced with the time elapsed since
you joined the lab, in years.  The supporting and manageareal roles
in the lab are:
\newcounter{supportlevels}
\begin{list}{S\arabic{supportlevels}-$x$:~}{\usecounter{supportlevels}}
\item \textbf{Administrative support staff.}\marginnote{Support-focused levels of the lab hierarchy}  This category includes
  lab members whose primary roles, regardless of their academic
  degree, are to provide administrative assistance (e.g.\ assisting with
  grant or paper submissions, registration, scheduling, coordination)
  to facilitate scientific research in the lab.
\item \textbf{Specialists.}  This category includes lab
  members who, regardless of their academic degree, bring
  a specific special scientific skill to the lab (e.g.\ programmers, graphic
  artists).  However, unlike research-focused positions, the primary
  role of specialist lab members is to provide research \textit{expertise}
  rather than determine research \textit{direction}.
\item \textbf{Project managers.}  This category includes lab members
  whose primary role is to provide managerial support to the lab by
  helping to direct and organize personnel and lab resources.
  Generally project managers will have earned a graduate degree.
\end{list}
In addition to the above components of the lab hierarchy, there's one more
lab position that plays an important role:
\newcounter{directorlevel}
\begin{list}{J\arabic{directorlevel}-$\infty$:~}{\usecounter{directorlevel}}
\item \textbf{Lab director.}  ``\textit{There can be only one...}''
  Becoming the lab director entails convincing a search committee
  at this or another department/university that you have what it takes to be a
  Supreme Ruler of Laboratory Science. Note: this position may or may not come with
  tenure and/or additional people to report to (e.g.\ senior
  colleagues, funding
  agencies, deans, provosts, presidents, etc.).  This position is both
  research-focused and support-focused, hence its own designation.
\end{list}

\newpage
\section{Project hierarchies}
Lab members at any level in the lab hierarchy\marginnote{The mapping
  between people and project roles is many-to-many: one person may play multiple project roles, and one
  project role may be shared by multiple people.} may play different
roles on different projects.
The role you play on a given project determines (for that project)
your research and administrative responsibilities, your supervisory
responsibilities, your role in decision making or strategizing, and
your authorship position in papers or conference presentations about
the project.  The project roles are:\marginnote{Authorship on project communications requires
    meeting ALL of the following criteria:\begin{itemize}
\item Substantial contributions to the conception or design of the
  work; or the acquisition, analysis, or interpretation of the data
\item Drafting the work or revising it critically for important
  intellectual content
\item Final approval of the version to be published
\item Agreement to be accountable for all apsects of the work in
  ensuring that questions related to the accuracy or integrity of any
  part of the work are appropriately investigated and resolved.
\end{itemize}
If a project collaborator does not meet all of the above criteria (but meets some of
them), they may be listed in the ``acknowledgements'' section at the
discresion of the lab director.}
 \newcounter{roles}
\begin{list}{P\arabic{roles}:~}{\usecounter{roles}}
\item \textbf{Project assistant.}  This is a minor ``non-authorship''
  role (generally project assitants will be listed in manuscript
  acknowledgements).  This role is assumed by all project participants
  who do not satisfy the criteria for paper
  authorship, but who directly
  participant in the project's conceptual or scientific development.
  An example might be an undergraduate research assistant who helps to
  run participants but who does not contribute to any analyses or to
  the itellectual development of the project.
\item \textbf{Junior co-author.}  This role is generally assumed by
  junior lab members (e.g.\ research assistants, junior grad
  students).  Junior co-authors assist with all aspects of carrying
  out the research (e.g.\ helping to write code, run participants,
  generate graphics, review communications, etc.).  The primary
  difference from a senior co-author role is that junior co-authors
  provde less conceptual guidance than senior co-authors.  Junior
  co-authors will generally hold ``middle author'' positions on
  communications related to the project.
\item \textbf{Senior co-author.}  This role is generally assumed by
  senior lab members (e.g.\ postdocs, senior grad students).  Senior
  co-authors provide conceptual guidance, research assistance (though
  to a lesser extent than a project lead), experimental assistance
  (though to a lesser extant than a project lead), and writing
  assistance (e.g.\ providing comments on manuscript drafts, posters,
  and/or slides).  Senior co-authors will generally hold ``middle
  author'' positions on communications related to the project.
\item \textbf{Project lead.}  This person is responsible for
  overseeing day-to-day operations for the project, and reports to the
  PI (or co-PI).  This person's responsibilities include: writing and
  running experiments (or directly overseeing these processes), coding
  and running analyses (or directly overseeing these processes),
  writing the first complete draft of the manuscript(s), editing the
  manuscript(s), submitting the manuscript(s), designing conference
  posters, and putting together slides for talks (which may be
  presented by the project lead or lab director).  The project lead is
  also responsible for scheduling project meetings and organizing all
  materials (code, data, etc.) related to the project.  The project
  lead will generally be the first author on communications
  (manuscripts, abstracts, posters, talks) related to
  the project.
\item \textbf{Principle investigator (PI).}  This is the person listed
  as the PI (or co-PI) on the grant that's funding the project.  (This
  role may be shared, e.g.\ in the case of a co-PI.)  Generally the
  (co-)PI will also be the project lead and/or will be the lab
  director.  The PI is responsible for overseeing the project's budget
  (in collaboration with the lab director) and ensuring that the
  project timeline is adhered to (in collaboration with the lab
  director).  The PI should discuss ``order of authorship'' on project
  communications with the lab director and ensure that all project
  members are informed of their order in the author sequence at the
  start of the project.  Prior to the project starting, the PI is also
  responsible for drafting and submitting the grant application (in
  collaboration with the lab director).
\item \textbf{Lab director.}  This is the most senior person on the
  project.  Generally this will be the lab director of the
  computational memory lab (Jeremy Manning).  For collaborative
  projects, the lab director role may fall to another faculty-level collaborator.
\end{list}

% Generally speaking, your immediate
% supervisor on a given project will be someone at a higher level of the
% hierarchy than you (and probably L4 or higher).  If you are looking
% for help with a particular technique, your best bet is to find someone
% who has been in the lab longer than you have, and who 

\newpage
\section{Starting a new project}
Our lab uses\marginnote{JEREMY TO DO: create a ``Getting started checklist''
  and add it to the lab manual.} a number of project management tools and policies to
promote continuity across projects and lab members.  When you start a
new project as a project lead, here are the steps you should
take:
\begin{enumerate}
\item Come up with a project name to uniquely identify your project
  (e.g.\ ``EEG mind decoder'').  Keep naming conventions
  consistent; the project will be referred to using this name from now on.
\item Create a
  \href{https://redbooth.com/}{RedBooth}\marginnote{\texttt{TASK:}
    Create a RedBooth account if you don't already have one.  Make
    sure you add yourself to the lab's group.} project and add
  all team members.  See \textit{Task management using Redbooth} for
  detailed instructions on how to organize the new Redbooth project.
\item Create a \href{https://bitbucket.org/}{Bitbucket}\marginnote{\texttt{TASK:}
    Create a Bitbucket account if you don't already have one.} repository for
  your project.  Give all team members write access to the repository;
  PIs and the lab director should have administrative access.  See
  \textit{Code and document management using Bitbucket} for detailed instructions on how to
  organize the Bitbucket repository.  Add a link to the Redbooth
  project in the project's Bitbucket wiki page.  Also add a link to
  the Bitbucket repository in Redbooth.
\item Create a proposed budget\marginnote{All equipment purchases or
    other lab charges must be approved in writing by Jeremy Manning
    \textit{prior to spending lab funds}.  IMPORTANT NOTE: Budget
    spreasheets should not be attached to the Redbooth project.
    All budgets are to be kept confidential within the lab.} spreadsheet (in
  \href{https://docs.google.com/}{Google Docs}) for project equipment
  purchases and subject payments.  Other project costs (travel
  expenses, publication fees, conference registration fees, textbooks,
  etc.) should be added to the spreadsheet as they become known.  Send
  a link to the spreadsheet to Jeremy and cc the project PI (if not
  Jeremy).
\item Create a \href{https://docs.google.com/}{Google Docs}\marginnote{\texttt{TASK:}
    Create a Google account if you don't already have one (or if you'd
    like to create a new one for lab-only use).  Make sure your
    account is linked with the lab's Google Apps account, which will
    give you access to extra storage space and some other additional functionality.}
  spreadsheet outlining all project personnel and their project roles,
  and share it with all team member (the PI may ask other project
  members to fill in personal information and contact details).  This should
  include each project member's:
\begin{itemize}
\item Full name
\item Lab hierarchy code (e.g.\ R4-0, J1, etc.)
\item Project hierarchy codes (e.g.\ P4 and P5)
\item Expected project role (keep it brief-- e.g.\ ``Project management; Code
  experiment; Run participants; Ensure consistent levels of caffenation
  maintained by all project members during lab business hours.'')
\item Proposed authorship order (this will apply to any project communication-- manuscripts,
  abstracts, talks, etc.)
\item Contact information (preferred email, preferred phone number, preferred mailing address,
  and primary office number or address if different from mailing
  address)
\item Acknowledgement that the project member has joined the project,
  looked over (and corrected as necessary) their information, 
  and accepts their project role.  This column must be filled out
  individually by each team member (each team member should type their
  initials into the project field).
\end{itemize}
Add a link to the personnel spreadsheet in \href{https://redbooth.com/}{RedBooth}.
\item Create a \href{https://calendar.google.com}{Google Calendar} for
  project meetings and events.
\item Add a line to your lab snippet template for this project.  (See
  \textit{Lab Snippets}.)
\item Create a new
  \href{https://groups.google.com/forum/#!overview}{Google Group} for
  the project and invite all team members.  Via Google Groups, send a polite but brief message to all project personnel welcoming them to the
  project.  Include the following information:
\begin{itemize}
\item A 1-2 sentence description of the project.
\item A link to a \href{http://www.doodle.com}{Doodle
    Poll} proposing some times for a first team meeting.
\item A request that all team members join the project in Redbooth and
  Bitbucket.  If any team members are not current lab members, also
  include a brief description of what Redbooth and Bitbucket are.  (A
  project management tool and version control system, respectively.)
\item A iCal link to the project's Google calendar.  Instructions for
  sharing calendars may be found \href{https://support.google.com/calendar/answer/37082?hl=en}{here}.
\item A note that you have outlined expected project roles and
  responsibilities in Redbooth, and that these roles and
  responsibilities will be discussed during the first project
  meeting.  Ask the project members to look over what you've proposed
  and make note of any comments, questions, or concerns (either within
  Redbooth or elsewhere) prior to that first meeting.
\end{itemize}
\end{enumerate}

%\newpage
\section{Joining a project}
If you are starting a new project (as the project lead), you
automatically ``join'' the project.  If you are a non-project lead
team member, your agreement to join the project is officially
signified when you add your initials to the project's Google Doc (sent
out by the project lead) and agree to the project responsibilities
outlined there.  All project communications should be sent via the
project's Google Groups mailing list.


\newpage
\section{Scheduling}
Complex dynamic systems can be difficult to understand (e.g.\
describe, compute with).  Fortunately for us, we do not need to start
entirely from scratch with respect to attempting to organize some
complex dynamic system we care about in our lab.  For example, we can
use tools like calendars and other software packages to organize and
understand our own temporal dynamics.  Our lab's scheduling policies
are intended to facilitate lab member interactions between ourselves,
our collaborators, and our experimental participants.  There are three
basic tools the lab uses to organize and schedule events:
\begin{itemize}
\item \href{http://calendar.google.com}{Google Calendar}.  We use the
  main lab calendar to keep track of lab-wide events including lab
  meetings, conferences, important talks.  We use the DHMC events
  calendar to keep track of important events and meetings at the
  DHMC.  We use the out-of-lab calendar to keep track of known
  absenses (e.g. illness, travel, holidays, etc.).  We use the
  meetings calendar to keep track of scheduled meetings (regularly
  scheduled meetings with Jeremy or senior lab members, project
  meetings, etc.).  When you add an event, it is important to include
  the following information as a comment (this does not apply to ``out-of-lab'' events):
\begin{itemize}
  \item Key contact names and contact information (email or phone)
  \item Physical address (where the event will take place)
  \item A brief description of the event and/or other relevant information
  \item Attach any relevant documents via Google Docs
\end{itemize}
\item \href{http://www.doodle.com}{Doodle}.  We use Doodle to converge
  on mutually good meeting times that fit (as well as possible) with
  everyone's busy schedules.
\item \href{http://www.groups.google.com}{Google Groups}.  We use
  Google Groups to send coordination-related emails to all relevant
  lab or project personnel.
\end{itemize}

 \subsection{Attendance policy}
 % As you move up in the lab hierarchy, our policy is to afford you
 % increasing scheduling flexibility (which, in turn, assumes increasing
 % responsibility on your part).  Increased \textit{scheduling
 %   flexibility} comes in the form of less frequent check-ins (e.g.\
 % times you are required to meet with your supervisor) and less
 % structured research time (e.g.\ your level of independence as a
 % researcher, as determined by your supervisor).  Increased
 % \textit{responsibility} comes in the form of increased expectations
 % placed on you as a researcher (in terms of research effort and
 % productivity).  
 We abide by a ``common sense'' attendance policy that relies on an
 honor system.  If you cannot attend a lab event or meeting, your
 privacy will be respected: you do not (generally speaking) need to
 provide a reason for your absense (although you are honor bound not
 to abuse this system!)-- but you are expected to responsibly manage
 your schedule so that you get your work done nad minimize
 inconvenience to others to the extent possible.  The one exception is
 that if you seem to be
 abusing this system (as determined by your supervisor), you may be
 asked to provide additional information (in a way that does not
 invade your privacy-- and if you are worried that this policy is overly
 intrusive, please bring your concerns to Jeremy).  Here's the official
 lab attendance policy:
\begin{itemize}
\item All absenses should be entered into the out-of-lab calendar.
  This should be done prior to your absense if possible; the
  out-of-lab calendar is used to track vacation time, sick leave, etc.
  If you will be out of the lab unexpectedly, it is your
  responsibility to notify your immediate supervisor (cc'ing Jeremy)
  as well as anyone else your absense will affect (e.g. people you're
  scheduled to meet with, etc.).  To enter an absense, simply create a
  new out-of-lab event with your first name followed by the word
  ``out.''  (E.g. if Jeremy is out of the lab he would create an event
  called ``Jeremy out''.)  Out-of-lab events can be all day events,
  multi-day events, or partial day events.  You do not need to (and
  should not) label your event (e.g.\ sick time, vacation time,
  conference time, etc.\ will be indistinguishable on the out-of-lab
  calendar); this is done to protect the privacy of all lab members.
\begin{itemize}
\item If you are feeling sick, \textit{do not come into the office}.  We can
  arrange virtual meetings (if you are feeling well enough) or
  re-schedule as needed.  The health and safety of the lab is the top
  priority.  Enter your expected ``out of lab'' duration into the
  out-of-lab calendar (update as you have more information so that we
  can correctly track your time).
\item If you must be out for an unexpected emergency, simply give as
  much notice as is possible.  Enter your expected ``out of lab''
  duration into the out-of-lab calendar (update as you have more
  information).
\item If you know that you will be out of the lab well in advance
  (e.g.\ classes, vacations, religious holidays, conferences, job interviews,
  doctor appointments, etc.) enter this into the out-of-lab calendar.  This should be done
  at least 2 weeks in advance, and preferably as far in advance as
  possible.  Enter this information into the out-of-lab calendar.
\end{itemize}
\item You are expected to attend all lab meetings and other regularly
  scheduled meetings that are directly relevant to your work in the
  lab.
\item If you are scheduled to present at a conference (i.e.\ you
  submit an abstract and the abstract is accepted as a talk or
  poster), you are expected to attend the conference to present your
  work.
\item You are strongly encouraged to attend relevant PBS talks, DHMC
  meetings and talks, thesis defenses, and other relevant lab and/or
  Dartmouth-affiliated events.  If you are overwhelmed with other
  work, have a conflicting meeting, are running an experimental
  participant, or are out of the lab for other reasons, you do not
  need to provide a reason for your absense (unless you've previously agreed to
  attend).
\item You are expected to tend to your out-of-lab responsibilities
  (classes, family duties, etc.) as a top priority.  If these conflict
  with your lab responsibilities please discuss with your supervisor
  and/or Jeremy as applicable.
\end{itemize}


 \section{Lab finances}
 As with most academic research labs, we (sadly!) must conduct our
 research within a limited research budget.  The lab's financial
 policy is the following: we will do whatever is possible to ensure
 you have the equipment and resources you \textit{need} to do your best work.
 If you can adequately justify an expense (typically done when proposing a
 project budget except when first starting out in the lab) and sufficient funds
 are available, then we will spend what it takes to get the job done.
 If you cannot justify an expense, or if the lab does not have
 sufficient funds, then we will likely need to get creative by
 figuring out how to get the job done anyway on a seemingly too-small
 budget.  Usually we'll find ourselves somewhere in the middle of this
 continuum, which will help us to stretch our limited budget as far as
 possible while not making ourselves crazy or losing too much
 productivity in the process.

\subsection{Startup money}
Startup money is \textit{unrestricted} money that may be used to
purchase equipment, hire personnel, pay travel expenses, fund lab
renovations, and grease the day-to-day wheels of the labs existance.
Critically, startup money may be used for any project, and is
therefore extremely valuable and special.  (Grant money is generally
designated for a specific project and may only be used for that one
project.)  Also, startup money never gets replenshed, which adds an
extra dimension of caution to the mix.

As a new lab member, you will get your own little share of the lab's
startup money.  \marginnote{Although you'll get some degree of flexibility with
respect to how you choose to spend this money (with more senior lab
members getting more flexibility), all purchase decisions must
still be approved by Jeremy (just as Jeremy's purchases must in turn
be approved by the administration, whose purchases must in turn be
approved by the senior administration, and so on).  Futher, all
equipment you buy is the property of the lab.  With the exception of
laptop computers, all equipment, books, etc. are expected to remain in
the lab.  If you leave the lab,
you are expected to return the lab's property!}  Your budget will
generally consist of:\marginnote{NOTE: ``Mini startup'' money may not be used
  to pay anyone's salary.}
\begin{itemize}
\item An intitial pool of money to be used to buy a computer (laptop,
  desktop, or both) plus any minor computer-related accessories you'll
  need (monitor, keyboard/mouse, etc.).  You may also be allocated an
  existing computer from the available lab stock in leu of a dollar amount.  Any computer
  equipment you need, repairs, or other computing costs you incur,
  will be expected to come from this pool.
\item A yearly travel budget.  Any lab-related travel you do will be
  expected to come from this pool.
\item An equipment budget.  Any experiment-specific equipment you buy
  will be expected to come from this pool.
\item A ``miscellaneous'' budget.  This is expected to cover things
  like textbooks, publication costs, poster printing, etc.
\end{itemize}

The precise amount of your mini startup fund will depend on:
\begin{itemize}
\item How long you expect
to stay in the lab.  For example, a short-term rotation or senior thesis
student will generally be allocated less money than a longer term
student, research assistant, or postdoctoral researcher.
\item What research you expect to do.  Some people need special
  equipment, money for subjects, etc., and may be
  allocated more money accordingly.
\item How much you are expected to travel.  If you're going to be
  going to lots of conferences, you'll need more money to get there!
\item How much you're expected to publish.
\end{itemize}

You will be allocated a modest (default) amount of money when you start
in the lab.  To get additional money, you can write a short (1-2 page)
mini ``grant'' proposal describing how you want to use the money.  You
may apply as often as you want and there is no limit to how much you
request (although in practice funding decisions will depend on lab
priorities, availability of funds, etc.).  You may appeal a decision by
submitting a revised proposal.

 \subsection{Applying for research and training grants}
All lab members are strongly encouraged to apply for research and/or
training grants.  Obtaining independent funding is how you become 

 \subsection{Equipment policy}
 \subsection{Computers}
 \subsection{Other research equipment}

 \subsection{Travel policy}

 \subsection{Poster printing}

 \subsection{Publication costs}

 \subsection{Subject payments}





 \section{Lab mailing list and contact information}

\section{Task management using Redbooth}

 \section{Code and document management using Bitbucket}\label{sec:bitbucket}
 We use \href{https://www.bitbucket.org/}{Bitbucket}\marginnote{The lab's standard practice is to use the
 Git version control system.  All projects \textit{must} be
 managed using Git and hosted on Bitbucket unless there is a
 compelling (and documented!) reason to use another system.} as a means of organizing, maintaining, and sharing
 our code and documents.  Bitbucket provides a unified framework for
 documenting changes over time, managing and integrating code from
 multiple users, and submitting bug reports or feature requests. 
Atlassian's
 \href{https://www.atlassian.com/software/sourcetree/overview?_mid=1ba3573dadf246f44f2a97bc50bfd72e&gclid=Cj0KEQiA4OqnBRDAj9aazvPji9ABEiQANq28oBGvgRhznXBb_RDL6QRe0IM7vvEUXFkRDoWBSbpmsmAaAkCE8P8HAQ}{SourceTree}
 client provides a convenient and easy-to-use desktop client for OS X
 and Windows systems.  Each project should have \textit{two} Bitbucket
 repositories.  The first repository ($<$projectname$>$-main) is used to
 manage experiment and analysis code and related (non-data) files.
 The second repository ($<$projectname$>$-communications) is used to
 manage documents and images (papers, posters, slides, etc.) related
 to \textit{communicating} the work to the broader scientific
 community.  This two repository system is intended to prevent the main project
 repository (which will eventually be shared publically in most cases)
 from (a) growing too large and (b) containing information or files
 that would not normally be shared publically (e.g.\ in-progress
 drafts of manuscripts, raw image files for posters, editable figures,
 etc.).  The \textit{main} repository should be organized using the following directory structure:
\begin{itemize}
\item \textit{code/main}: contains code for running experiments and
  analyses.  When the paper is published, this folder (and/or its contents) will be made
  publically available.  How this folder gets organized is up to the
  project lead.
\item \textit{code/main/contents.txt}: a list of 
\item \textit{code/main/README.txt}: quick start instructions for
  using the code.
\item \textit{code/dev}: contains under-development and/or half-baked
  code.  All project-related code should start life in this folder,
  and should be moved to code/main after it has been debugged and
  after its unit tests have been written.
\item \textit{code/debug}: contains unit tests for code in code/main.
  Unit tests should provide example scripts for running code/main code
  along with specific detailed descriptions of the expected behavior
  or output.
\item \textit{docs/admin}: contains organizational files related to the
  project, including:
\begin{itemize}
\item \textbf{Project vision.}  A high level description of the
  project along with a (rough) proposed timeline for the project's
  phases.
\item \textbf{Personnel list.}  A description of all project members
  and their expected roles on the project.  Authorship order must be
  specified in this document, and changes must be approved by the lab
  director.
\end{itemize}
\item \textit{docs/admin/forms}: official forms and approvals related to
  the project.  Each experiment should have its own sub-folder
  containing:
\begin{itemize}
  \item IRB approval forms
  \item Participant consent forms
  \item Participant exit surveys (if applicable)
  \item Participant instructions
  \item Participant debriefing document
  \item Ammendments to any forms (e.g. IRB approval extensions)
  \item A document listing budget codes for grants that fund any part
    of the project  
\end{itemize}
\end{itemize}

The \textit{communications} repository should be organized using the following directory structure:
\begin{itemize}
\item \textit{papers}: each paper should exist as a sub-folder of
  this directory.  Each paper's sub-folder should contain the following:
\begin{itemize}
\item \textit{paper\_name.tex}: main source file for document text
\item \textit{boneyard.tex}: a source file containing pasted snippets of
  discarded text that might conceivably be useful at some later point
  (in this or other projects).
\item Other files needed to compile the main source file into a
  complete PDF.
\item \textit{figs}: sub-folder containing complete PDF figures.
\item \textit{figs/source}: sub-folder containing files relevant to
  constructing each manuscript figure (e.g. the images for each panel
  in a figure).  Each figure should have its own sub-folder.  For compound figures (e.g. a panel that has
  multiple sub-panels), each panel should have its own \textit{source}
  directory that lives in its parent figure's source directory.  Raw
  data should not be stored in these folders, but intermediate files
  or images that would be time consuming to regenerate (or that are
  useful to have in an easily accessed location) should be included.
  In addition, Adobe source files (e.g.\ .ai or .psd files) should be
  included in the source folders.
\item \textit{replicate}: sub-folder containing links to the data and code
  relevant to the paper, along with \textbf{detailed} helpful
  instructions that would allow someone to download the project's \textit{main}
  Bitbucket repository and data, and reproduce figures from each paper
  from scratch (assuming they had the appropriate hardware to run the
  analyses).  The instructions should also list system requirements
  (as applicable) and software dependencies (along with download links
  and installation instructions).  This folder should also contain a
  ``make\_figs'' script that, when run, should reproduce the basic 
  versions of the paper's figures (e.g.\ before editing fonts and line
  styles or composing figure panels in Illustrator), assuming the
  \textit{main} project repository has been downloaded to the specified location.
\end{itemize}
\item \textit{posters}: each poster should exist as a sub-folder
  of this directory.  Each poster's sub-folder should contain:
\begin{itemize}
  \item The main .ai, .pdf, .pptx, or .tex poster file.
  \item A copy of the associated conference abstract or paper.
  \item A README.txt file containing bibliographic information for the
    poster.
\end{itemize}
\item \textit{slides}: sub-folder containing the .key, .pptx, or
  .pdf files for each presentation.
\item \textit{source}: sub-folder containing all source
  materials (figures, etc.) linked to in any poster or presentation.  The source
  directory may contain additional sub-folders, but does not need
  to be separated by poster or presentation.  This is because oftentimes the
  same source file(s) will be shared across multiple posters or presentations.
\end{itemize}

 \section{Data repository}
 \subsection{Lab Dropbox account}

 \section{Internal Review Board (IRB) approval process}
 \subsection{List of active protocols}
 \subsection{List of inactive protocols}


 \chapter{Official lab techniques}\label{ch:techniques}
 \section{Document formatting using \LaTeX}
 \section{Coding experiments using PsychToolbox}
 \section{Data analysis using MATLAB and Python}



 \section{Behavioral experiments}

 \section{fMRI}
 \subsection{Running fMRI participants}
 \subsection{Data preprocessing}
 \subsection{Multivariate Pattern Analysis}
 \subsection{Topographic Factor Analysis}

 \section{Scalp EEG and ECoG}
 \subsection{Running EEG participants}
 \subsection{Running hospital patients (ECoG)}

 \section{Cluster computing}


% Citation example \cite{Tufte2001}, notice how the citation is in the margin. This is an example of how to add something to the index at the end of the document.\index{citation}

% \newthought{Example of} the \texttt{newthought} command for starting new sections. Typography examples: \allcaps{all caps} and \smallcaps{small caps}.

%------------------------------------------------

% \section{Figures}

% \lipsum[1] 

% \begin{marginfigure}
% \includegraphics[width=\linewidth]{helix}
% \caption{This is a margin figure. The helix is defined by $x = \cos(2\pi z)$, $y = \sin(2\pi z)$, and $z = [0, 2.7]$. The figure was drawn using Asymptote (\url{http://asymptote.sf.net/}).}
% \label{fig:marginfig}
% \end{marginfigure}

% \lipsum[2]

% \begin{figure*}[h]
% \includegraphics[width=\linewidth]{sine.pdf}
% \caption{This graph shows $y = \sin x$ from about $x = [-10, 10]$.
% \emph{Notice that this figure takes up the full page width.}}
% \label{fig:fullfig}
% \end{figure*}

% \lipsum[3]

% %------------------------------------------------

% \section{Tables} \marginnote{This is a random margin note. Notice that there isn't a number preceding the note, and there is no number in the main text where this note was written. Use \texttt{sidenote} to use a number.}

% \lipsum[4]

% \begin{table} % Add the following just after the closing bracket on this line to specify a position for the table on the page: [h], [t], [b] or [p] - these mean: here, top, bottom and on a separate page, respectively
% \centering % Centers the table on the page, comment out to left-justify
% \begin{tabular}{l c c c c c} % The final bracket specifies the number of columns in the table along with left and right borders which are specified using vertical bars (|); each column can be left, right or center-justified using l, r or c. To specify a precise width, use p{width}, e.g. p{5cm}
% \toprule % Top horizontal line
% & \multicolumn{5}{c}{Growth Media} \\ % Amalgamating several columns into one cell is done using the \multicolumn command as seen on this line
% \cmidrule(l){2-6} % Horizontal line spanning less than the full width of the table - you can add (r) or (l) just before the opening curly bracket to shorten the rule on the left or right side
% Strain & 1 & 2 & 3 & 4 & 5\\ % Column names row
% \midrule % In-table horizontal line
% GDS1002 & 0.962 & 0.821 & 0.356 & 0.682 & 0.801\\ % Content row 1
% NWN652 & 0.981 & 0.891 & 0.527 & 0.574 & 0.984\\ % Content row 2
% PPD234 & 0.915 & 0.936 & 0.491 & 0.276 & 0.965\\ % Content row 3
% JSB126 & 0.828 & 0.827 & 0.528 & 0.518 & 0.926\\ % Content row 4
% JSB724 & 0.916 & 0.933 & 0.482 & 0.644 & 0.937\\ % Content row 5
% \midrule % In-table horizontal line
% \midrule % In-table horizontal line
% Average Rate & 0.920 & 0.882 & 0.477 & 0.539 & 0.923\\ % Summary/total row
% \bottomrule % Bottom horizontal line
% \end{tabular}
% \caption{Table caption text} % Table caption, can be commented out if no caption is required
% \label{tab:template} % A label for referencing this table elsewhere, references are used in text as \ref{label}
% \end{table}

%----------------------------------------------------------------------------------------

\mainmatter

%----------------------------------------------------------------------------------------
%	CHAPTER 1
%----------------------------------------------------------------------------------------

%\chapter{Chapter 1 Title}
%\label{ch:1}

%------------------------------------------------

% \section{Section 1 - Fullwidth Environment Example}

% \begin{fullwidth}
% \lipsum[5]
% \end{fullwidth}

% \subsection{Subsection 1}

% \lipsum[6-7]

% \subsection{Subsection 2}

% \lipsum[7-8]



%----------------------------------------------------------------------------------------

\backmatter

%----------------------------------------------------------------------------------------
%	BIBLIOGRAPHY
%----------------------------------------------------------------------------------------

\bibliography{bibliography} % Use the bibliography.bib file for the bibliography
\bibliographystyle{plainnat} % Use the plainnat style of referencing

%----------------------------------------------------------------------------------------

\printindex % Print the index at the very end of the document

\end{document}